\documentclass[11pt,twoside]{article}

\usepackage[top=3cm,bottom=2.5cm,left=3cm,right=2cm]{geometry}
\setlength{\parindent}{0pt}

\begin{document}

\title{GIT-SVN basic commands}
\maketitle

\subsection*{Important remark}

It might be useful to issue some git-svn commands beforehand with the \verb|-n| option (or with its longer name \verb|--dry-run|).
This option shows what will be impacted by the \texttt{git-svn} command before issuing the actual command.
This can be used with the \texttt{dcommit}, \texttt{rebase}, \texttt{branch} and \texttt{tag} commands.

\subsection*{Git-svn -- Initial checkout}
\begin{verbatim}
  $ git svn clone -s svn+ssh://svn@geodynamics.org/cig/seismo/3D/SPECFEM3D_GLOBE
\end{verbatim}
The previous command checks out the full svn history and converts it into a git history. It takes a huge amount of time. If the full history is not needed, one can check out a partial history.
For instance, checking out revisions from $20\:000$ to the last one (\texttt{HEAD}) is done with:
\begin{verbatim}
  $ git svn clone -s -r 20000:HEAD \
      svn+ssh://svn@geodynamics.org/cig/seismo/3D/SPECFEM3D_GLOBE
\end{verbatim}

\subsection*{Git-svn -- SVN branches}

Taking a look at the different branches available:
\begin{verbatim}
  $ git branch -a
  * master
    remotes/NOISE_TOMOGRAPHY
    remotes/ORIGINAL
    remotes/SPECFEM3D_GLOBE_SUNFLOWER
    remotes/adjoint
    remotes/pluggable
    remotes/tags/v5.1.5
    remotes/trunk
\end{verbatim}

Creating a new branch \texttt{SPECFEM3D\_GLOBE\_ADIOS} on the SVN server:
\begin{verbatim}
  $ git svn branch SPECFEM3D_GLOBE_ADIOS
\end{verbatim}

Switching to a remote branch. It is mandatory, else the commit will be done in the master branch
\begin{verbatim}
  $ git checkout -b local/SPECFEM3D_GLOBE_ADIOS SPECFEM3D_GLOBE_ADIOS
\end{verbatim}

\subsection*{Git -- Checking the status of the file}
Before doing any modification it is mandatory to know which files are tracked, have been modified or deleted.

\begin{verbatim}
  $ git status
\end{verbatim}

If some files appear to be untracked they can be added to the local git repository with:

\begin{verbatim}
  $ git add filename

  $ git status
  # should now return the added file
  # as `new file' instead of `untracked'.
\end{verbatim}

If some files were modified, before committing the modifications, it may be useful to see what has changed :

\begin{verbatim}
  $ git diff filename
\end{verbatim}

\subsection*{Git -- Committing changes into the local repository}

\begin{verbatim}
  $ git commit
\end{verbatim}

This command will open your favorite text editor (it is likely to be vi without any particular configuration) and ask you for a comment about the commit. If nothing is written, the commit will abort.

\subsection*{Git -- Removing files}

\begin{verbatim}
  $ git rm filename
  # prepare to remove the file from the git local repository
  # and remove the file from your workspace.

  $ git commit #...
\end{verbatim}

\subsection*{Git -- Moving files}

\begin{verbatim}
  $ git mv old_file new_file

  $ git commit
\end{verbatim}

OR

\begin{verbatim}
  $ mv old_file new_file

  $ git rm old_file

  $ git add new_file

  $ git commit
\end{verbatim}

\subsection*{Git -- Commit history}

\begin{verbatim}
  $ git log
\end{verbatim}

\subsection*{Git -- Changing the last commit}

Do not do this on commits that have already been sent to the SVN server.

\begin{verbatim}
  $ git commit -m 'initial commit'

  $ git add forgotten_file

  $ git commit --amend
\end{verbatim}

\subsection*{Git -- Unstaging a staged file (i.e. Removing a file from the list of files to be
committed)}

\begin{verbatim}
  $ git status
  # the file has to be in the `changes to be committed' section

  $ git reset HEAD filename
  filename: locally modified

  $ git status
  # the file should be in the `Changes not staged for commit' section
\end{verbatim}

\subsection*{Git -- Unmodifying a Modified File}

\begin{verbatim}
  $ git checkout -- filename

  $ git status
\end{verbatim}

\subsection*{Git-svn -- Back to the SVN server}

Push modifications up to the SVN server:
\begin{verbatim}
  $ git svn dcommit
\end{verbatim}
That will also bring your local tree up to date. To bring your tree up to date in general, run:
\begin{verbatim}
  $ git svn rebase
\end{verbatim}
This will update your local checkout and then re-apply your local unsubmitted commits on top of the new trunk so that the history remains linear.
If you want to get the commits for all branches that exist in your clone (without modifying local changes):
\begin{verbatim}
  $ git svn fetch
\end{verbatim}

\subsection*{Git -- Stashing changes temporarily}

If you have changes that have not been committed (locally), you will run into errors when you try to rebase from the SVN server.
You can temporarily stash your changes before updating the tree and then restore them afterwards.

\begin{verbatim}
  $ git stash
  # Changes to workspace from previous commit will be stored in a temporary area.

  $ git svn rebase
  # Do remote-to-local tree update.

  $ git stash pop
  # Previous changes are restored to workspace and deleted from temporary area.
\end{verbatim}

\subsection*{References and useful readings}

\begin{itemize}
\item \verb|http://git-scm.com/book|
\item \verb|http://git.or.cz/course/svn.html|
\item \verb|http://trac.parrot.org/parrot/wiki/git-svn-tutorial|
\end{itemize}

\end{document}

