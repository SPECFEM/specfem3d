\addcontentsline{toc}{chapter}{Copyright}

%%%%%%%%%%%%%%%%%%%%%%%%%%%%%%%%%%%%%%%%%%%%%%%%%

\chapter*{Copyright}

Main historical authors: Dimitri Komatitsch and Jeroen Tromp

Princeton University, USA, and CNRS / University of Marseille, France

$\copyright$ Princeton University and CNRS / University of Marseille, July 2012

This program is free software; you can redistribute it and/or modify
it under the terms of the GNU General Public License as published
by the Free Software Foundation (see Appendix \ref{cha:License}).\\


\textbf{\underline{Evolution of the code:}}\\

MPI v. 3.0, December 2014: many developers.
Convolutional PML, LDDRK time scheme, bulk attenuation support, simultaneous MPI runs,
ADIOS file I/O support, coupling with external codes, new seismogram names,
Deville routines for additional GLL degrees, tomography tools, unit/regression test framework,
improved CUDA GPUs performance, additonal GEOCUBIT support, better make compilation,
git versioning system. \\


MPI v. 2.1, July 2012: Max Rietmann, Peter Messmer, Daniel Peter, Dimitri
Komatitsch, Joseph Charles, Zhinan Xie: support for CUDA GPUs, better
CFL stability for the Stacey absorbing conditions. \\


MPI v. 2.0, November 2010: Dimitri Komatitsch, Nicolas Le Goff, Roland
Martin and Pieyre Le Loher, University of Pau, France, Daniel Peter,
Jeroen Tromp and the Princeton group of developers, Princeton University,
USA, and Emanuele Casarotti, INGV Roma, Italy: support for CUBIT meshes
decomposed by SCOTCH; much faster solver using Michel Deville's inlined
matrix products.\\


MPI v. 1.4 Dimitri Komatitsch, University of Pau, Qinya Liu and others,
Caltech, September 2006: better adjoint and kernel calculations, faster
and better I/Os on very large systems, many small improvements and
bug fixes.\\


MPI v. 1.3 Dimitri Komatitsch, University of Pau, and Qinya Liu, Caltech,
July 2005: serial version, regular mesh, adjoint and kernel calculations,
ParaView support.\\


MPI v. 1.2 Min Chen and Dimitri Komatitsch, Caltech, July 2004: full
anisotropy, volume movie.\\


MPI v. 1.1 Dimitri Komatitsch, Caltech, October 2002: Zhu's Moho map,
scaling of $V_{s}$ with depth, Hauksson's regional model, attenuation,
oceans, movies.\\


MPI v. 1.0 Dimitri Komatitsch, Caltech, USA, May 2002: first MPI version
based on global code.\\


Dimitri Komatitsch, IPG Paris, France, December 1996: first 3-D solver
for the CM-5 Connection Machine, parallelized on 128 processors using
Connection Machine Fortran.\\

