SPECFEM3D has the capability to compute the perturbations of gravity
induced by seismic waves at arbitrary locations above the Earth's surface.
The current file documents the usage of this feature and the related code
modifications.


Theory
------

The computation of gravity perturbations induced by deformation is based on
equation 4 of:
J. Harms, R. DeSalvo, S. Dorsher and V. Mandic (2009), "Simulation of underground
gravity gradients from stochastic seismic fields", Phys. Rev. D, 80, 122001


Enabling gravity computations
-----------------------------

Gravity perturbation can be computed in any SPECFEM3D simulation by
placing a file called "gravity_stations" in the DATA directory, in
addition to the regular input files.


Input file format
-----------------

The format of the "gravity_stations" input file is:

n dt_gap
x1  y1  z1
x2  y2  z2
... ... ...
xn  yn  zn

where
 n : number of stations where gravity time series are needed
 dt_gap : gravity time series are sampled every dt_gap time steps of the
          SPECFEM3D simulation


Output file format
-----------------

Time series of gravity are output in files named "OUTPUT_FILES/stat*.grav",
where * is the station index (one file per station). Their format is four
columns:
  t  ax  ay  az
(time and acceleration along x, y and z, respectively).


Code modifications
------------------

All the routines related to the gravity perturbation computations are
placed in one module called specfem3d/gravity_perturbation.f90. The module
provides three public subroutines and a flag. Each subroutine is invoked
in one of the following stages:
 1. during the initialization,
 2. during the iterative time stepping scheme and
 3. at the output stage.
In #1 the code checks for the presence of the input file "gravity_stations".
If the file exists, the flag "GRAVITY_SIMULATION" is turned on and the
subroutines #2 and #3 are invoked.


Author
------

Surendra Somala (Caltech) surendra@caltech.edu - 2013
with advice from Jan Harms and Pablo Ampuero (ampuero@gps.caltech.edu)
