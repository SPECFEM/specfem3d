\chapter{Adjoint Simulations}\label{cha:Adjoint-Simulations}

Adjoint simulations are generally performed for two distinct applications.
First, they can be used in point source moment-tensor inversions, or source imaging for
earthquakes with large ruptures such as the Lander's earthquake \citep{WaHe94}.
Second, they can be used to generate finite-frequency sensitivity
kernels that are a critical part of tomographic inversions based upon
3D reference models \citep{TrTaLi05,LiTr06,TrKoLi08,LiTr08}. In either
case, source parameter or velocity structure updates are sought to
minimize a specific misfit function (e.g., waveform or traveltime
differences), and the adjoint simulation provides a means of computing
the gradient of the misfit function and further reducing it in successive
iterations. Applications and procedures pertaining to source studies
and finite-frequency kernels are discussed in Sections~\ref{sec:Adjoint-simulation-sources}
and \ref{sec:Adjoint-simulation-finite}, respectively. The two related
parameters in the \texttt{Par\_file} are \texttt{SIMULATION\_TYPE}
(1, 2 or 3) and the \texttt{SAVE\_FORWARD} (boolean).


\section{Adjoint Simulations for Sources Only (not for the Model)}\label{sec:Adjoint-simulation-sources}

When a specific misfit function between data and synthetics is minimized to invert
for earthquake source parameters, the gradient of the misfit function
with respect to these source parameters can be computed by placing
time-reversed seismograms at the receivers as virtual sources
in an adjoint simulation. Then the value of the gradient is obtained
from the adjoint seismograms recorded at the original earthquake location.
\begin{enumerate}
\item \textbf{Prepare the adjoint sources} \label{enu:Prepare-the-adjoint}

\begin{enumerate}
\item First, run a regular forward simlation (\texttt{SIMULATION\_TYPE =
1} and \texttt{SAVE\_FORWARD = .false.}). You can automatically set
these two variables using the \texttt{\small utils/change\_simulation\_type.pl}{\small{}
script:}{\small \par}

\begin{lyxcode}
utils/change\_simulation\_type.pl~-f~
\end{lyxcode}

and then collect the recorded seismograms at all the stations given
in \texttt{DATA/STATIONS}.

\item Then select the stations for which you want to compute the time-reversed
adjoint sources and run the adjoint simulation, and compile them into
the \texttt{DATA/STATIONS\_ADJOINT} file, which has the same format
as the regular \texttt{DATA/STATIONS} file.

\begin{itemize}
\item Depending on what type of misfit function is used for the source inversion,
adjoint sources need to be computed from the original recorded seismograms
for the selected stations and saved in a sub-directory called \texttt{SEM/}
in the root directory of the code, with the format \texttt{NT.STA.BX?.adj}, where \texttt{NT},
\texttt{STA} are the network code and station name given in the \texttt{DATA/STATIONS\_ADJOINT}
file, and \texttt{BX?} represents the component name of a particular
adjoint seismogram. Please note that the band code can change depending
on your sampling rate (see Appendix~\ref{cha:channel-codes} for
further details).
\item The adjoint seismograms are in the same format as the original seismogram
(\texttt{NT.STA.BX?.sem?}), with the same start time, time interval
and record length.
\end{itemize}
\item Notice that even if you choose to time reverse only one component
from one specific station, you still need to supply all three components
because the code is expecting them (you can set the other two components
to be zero).
\item Also note that since time-reversal is done in the code itself, no
explicit time-reversing is needed for the preparation of the adjoint
sources, i.e., the adjoint sources are in the same forward time sense
as the original recorded seismograms.
\end{enumerate}
\item \textbf{Set the related parameters and run the adjoint simulation}\\
 In the \texttt{DATA/Par\_file}, set the two related parameters to
be \texttt{SIMULATION\_TYPE = 2} and \texttt{SAVE\_FORWARD = .false.}.
More conveniently, use the scripts \texttt{utils/change\_simulation\_type.pl}
to modify the \texttt{Par\_file} automatically (\texttt{change\_simulation\_type.pl
-a}). Then run the solver to launch the adjoint simulation.
\item \textbf{Collect the seismograms at the original source location}


After the adjoint simulation has completed successfully, collect the
seismograms from \texttt{LOCAL\_PATH}.
\begin{itemize}
\item These adjoint seismograms are recorded at the locations of the original
earthquake sources given by the \texttt{DATA/CMTSOLUTION} file, and
have names of the form \texttt{NT.S?????.S??.sem} for the six-component
strain tensor (\texttt{SNN,SEE,SZZ,SNE,SNZ,SEZ}) at these locations,
and ~\\
 \texttt{NT.S?????.BX?.sem} for the three-component displacements
(\texttt{BXN,BXE,BXZ}) recorded at these locations.
\item \texttt{S?????} denotes the source number; for example, if the original
\texttt{CMTSOLUTION} provides only a point source, then the seismograms
collected will start with \texttt{S00001}.
\item These adjoint seismograms provide critical information for the computation
of the gradient of the misfit function.
\end{itemize}
\end{enumerate}



\section{Adjoint Simulations for Finite-Frequency Kernels (Kernel Simulation)}\label{sec:Adjoint-simulation-finite}

Finite-frequency sensitivity kernels are computed in two successive
simulations (please refer to \citet{LiTr06} and \citet{TrKoLi08}
for details).
\begin{enumerate}
\item \textbf{Run a forward simulation with the state variables saved at
the end of the simulation}


Prepare the \texttt{\small CMTSOLUTION}{\small{} and }\texttt{\small STATIONS}{\small{}
files, set the parameters }\texttt{\small SIMULATION\_TYPE}{\small {}
}\texttt{\small =}{\small {} }\texttt{\small 1}{\small{} and }\texttt{\small SAVE\_FORWARD
=}{\small {} }\texttt{\small .true.}{\small{} in the }\texttt{\small Par\_file}{\small{}
(}\texttt{\small change\_simulation\_type -F}{\small ), and run the
solver.}{\small \par}
\begin{itemize}
\item Notice that attenuation is not implemented yet for the computation
of finite-frequency kernels; therefore set \texttt{ATTENUATION = .false.}
in the \texttt{Par\_file}.
\item We also suggest you modify the half duration of the \texttt{CMTSOLUTION}
to be similar to the accuracy of the simulation (see Equation \ref{eq:shortest_period})
to avoid too much high-frequency noise in the forward wavefield, although
theoretically the high-frequency noise should be eliminated when convolved
with an adjoint wavefield with the proper frequency content.
\item This forward simulation differs from the regular simulations (\texttt{\small SIMULATION\_TYPE}{\small {}
}\texttt{\small =}{\small {} }\texttt{\small 1}{\small{} and }\texttt{\small SAVE\_FORWARD}{\small {}
}\texttt{\small =}{\small {} }\texttt{\small .false.}{\small ) described
in the previous chapters in that the state variables for the last
time step of the simulation, including wavefields of the displacement,
velocity, acceleration, etc., are saved to the }\texttt{\small LOCAL\_PATH}{\small{}
to be used for the subsequent simulation. }{\small \par}
\item For regional simulations, the files recording the absorbing boundary
contribution are also written to the \texttt{LOCAL\_PATH} when \texttt{SAVE\_FORWARD
= .true.}.
\end{itemize}
\item \textbf{Prepare the adjoint sources}


The adjoint sources need to be prepared the same way as described
in the Section \ref{enu:Prepare-the-adjoint}.
\begin{itemize}
\item In the case of travel-time finite-frequency kernel for one source-receiver
pair, i.e., point source from the \texttt{CMTSOLUTION}, and one station
in the \texttt{STATIONS\_ADJOINT} list, we supply a sample program
in \texttt{utils/adjoint\_sources/traveltime/xcreate\_adjsrc\_traveltime}
to cut a certain portion of the original displacement seismograms
and convert it into the proper adjoint source to compute the finite-frequency
kernel.

\begin{lyxcode}
xcreate\_adjsrc\_traveltime~t1~t2~ifile{[}0-5{]}~E/N/Z-ascii-files~{[}baz{]}
\end{lyxcode}

where \texttt{t1} and \texttt{t2} are the start and end time of the
portion you are interested in, \texttt{ifile} denotes the component
of the seismograms to be used (0 for all three components, 1 for East,
2 for North, and 3 for vertical, 4 for transverse, and 5 for radial
component), \texttt{E/N/Z-ascii-files} indicate the three-component
displacement seismograms in the right order, and \texttt{baz} is the
back-azimuth of the station. Note that \texttt{baz} is only supplied
when \texttt{ifile} = 4 or 5.

\item Similarly, a sample program to compute adjoint sources for amplitude
finite-frequency kernels may be found in \texttt{utils/adjoint\_sources/amplitude}
and used in the same way as described for traveltime measurements

\begin{lyxcode}
xcreate\_adjsrc\_amplitude~t1~t2~ifile{[}0-5{]}~E/N/Z-ascii-files~{[}baz{]}.
\end{lyxcode}
\end{itemize}
\item \textbf{Run the kernel simulation}


With the successful forward simulation and the adjoint source ready
in the \texttt{SEM/} directory, set \texttt{SIMULATION\_TYPE
= 3} and \texttt{SAVE\_FORWARD = .false.} in the \texttt{Par\_file}
(you can use \texttt{change\_simulation\_type.pl -b}), and rerun the
solver.
\begin{itemize}
\item The adjoint simulation is launched together with the back reconstruction
of the original forward wavefield from the state variables saved from
the previous forward simulation, and the finite-frequency kernels
are computed by the interaction of the reconstructed forward wavefield
and the adjoint wavefield.
\item The back-reconstructed seismograms at the original station locations
are saved to the \texttt{LOCAL\_PATH} at the end of the kernel simulations,
and can be collected to the local disk.
\item These back-constructed seismograms can be compared with the time-reversed
original seismograms to assess the accuracy of the backward reconstruction,
and they should match very well.
\item The arrays for density, P-wave speed and S-wave speed kernels are
also saved in the \texttt{LOCAL\_PATH} with the names \texttt{proc??????\_rho(alpha,beta)\_kernel.bin},
where \texttt{proc??????} represents the processor number, \texttt{rho(alpha,beta)}
are the different types of kernels.
\end{itemize}
\item \textbf{Run the anisotropic kernel simulation}


Instead of the kernels for the isotropic wave speeds, you can also
compute the kernels for the 21 independent components $C_{IJ},\, I,J=1,...,6$
(using Voigt's notation) of the elastic tensor in the cartesian coordinate
system. This is done by setting \texttt{ANISOTROPIC\_KL} \texttt{=}
\texttt{.true.} in \texttt{constants.h} before compiling the package.
The definition of the parameters $C_{IJ}$ in terms of the corresponding
components $c_{ijkl},ijkl,i,j,k,l=1,2,3$ of the elastic tensor in
cartesian coordinates follows \citet{ChTr07}. The 21 anisotropic
kernels are saved in the \texttt{LOCAL\_PATH} in one file with the
name of \texttt{proc??????\_cijkl\_kernel.bin} (with \texttt{proc??????}
the processor number). The output kernels correspond to absolute perturbations
$\delta C_{IJ}$ of the elastic parameters and their unit is in $s/GPa/km^{3}$.
For consistency, the output density kernels with this option turned
on are for a perturbation $\delta\rho$ (and not $\frac{\delta\rho}{\rho}$)
and their unit is in s / (kg/m$^{3}$) / km$^{3}$. These `primary'
anisotropic kernels can then be combined to obtain the kernels for
different parameterizations of anisotropy. This can be done, for example,
when combining the kernel files from slices into one mesh file (see
Section~\ref{sec:Finite-Frequency-Kernels}).


If \texttt{ANISOTROPIC\_KL} \texttt{=} \texttt{.true.} by additionally
setting \texttt{ANISOTROPIC\_KL} \texttt{=} \texttt{.true.} in \texttt{constants.h}
the package will save anisotropic kernels paramterized as velocities
related to transverse isotropy based on the the Chen and Tromp parameters
\citet{ChTr07}. The kernels are saved as relative perturbations for
horizontal and vertical P and S velocities, $\alpha_{v},\alpha_{h},\beta_{v},\beta_{h}$.
Explicit relations can be found in appendix B. of \citet{SiLiTrTr07b}


The anisotropic kernels are only currently available for CPU mode.

\end{enumerate}
In general, the three steps need to be run sequentially to assure
proper access to the necessary files. If the simulations are run through
some cluster scheduling system (e.g., LSF), and the forward simulation
and the subsequent kernel simulations cannot be assigned to the same
set of computer nodes, the kernel simulation will not be able to access
the database files saved by the forward simulation. Solutions for
this dilemma are provided in Chapter~\ref{cha:Scheduler}. Visualization
of the finite-frequency kernels is discussed in Section~\ref{sec:Finite-Frequency-Kernels}.

