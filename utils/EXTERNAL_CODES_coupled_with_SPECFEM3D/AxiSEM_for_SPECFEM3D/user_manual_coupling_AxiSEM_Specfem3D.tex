% Type de document
\documentclass[a4paper,11pt]{article}

% Chargement des extensions
\usepackage[latin1]{inputenc}
 \usepackage{graphicx}
\begin{document}
{\centerline{\bf{Coupling AxiSEM and Specfem3D}}}
{\bf{Authors:}} V. Monteiller

\section{Meshfem3D}

running Meshfem3D with parameters : \newline

In the directory MESH/ \newline

ParFileMeshChunk  : parameters for the chunk meshing \newline

iasp91\_dsm or prem\_dsm : background model used in both AxiSEM ans Specfem3D \newline


In the Directory DATA/  \newline

CMTSOLUTION  : with 0 in order to not use source inside \newline

coeff\_poly\_deg12 : to generate smooth intitial solution  \newline

STATIONS  : stations files  \newline

Par\_file : with \newline

SIMULATION\_TYPE=1 \newline

SAVE\_FORWARD = .false. \newline

COUPLE\_WITH\_EXTERNAL\_CODE = .true \newline

EXTERNAL\_CODE\_TYPE    = 2 \newline

\section{AxiSEM mesher}

running AxiSEM mesher

\section{AxiSEM solver}
copy 2 files produced by meshfem3D in the running AxisSEM directory: \newline

input\_box.txt and input\_box\_sem\_cart.txt  \newline

add one first line to indicate the number of line to be read \newline

running AxiSEM solver with parameters (differences with *.TEMPLATES provided
bny AxiSEM) : \newline

inparam\_basic : \newline

ATTENUATION         false \newline
SAVE\_SNAPSHOTS     true \newline

inparam\_advanced : \newline

KERNEL\_WAVEFIELDS   true \newline
KERNEL\_IBEG         0 \newline
KERNEL\_IEND         4 \newline


\section{Specfem Partitionning}

running scotch partitionning

\section{Specfem Generate database}

running generate database

\section{Interface : expand 2D to 3D}

mpi run with arbitrary number of processes  \newline

parameter file : expand\_2D\_3D.par (in SOLVER directory)\newline

input\_box.txt \newline
input\_box\_sem\_cart.txt \newline
8                          \# number of AxiSEM mpi processes used in solver  \newline
60. 0.                     \# source position (lat lon)  \newline
0.  60.                    \# chnuk center (lat lon)    \newline
1                          \# number of axisem simus depends on moment tensor
used  \newline
8                          \# number of Specfem3D MPI processes   \newline
../../run\_synth\_alps/create\_mesh/MESH \# Specfem MESH directory  \newline
../../run\_synth\_alps/create\_mesh/OUTPUT\_FILES/DATABASES\_MPI \# Specfem

TRACTION DIRECTORY \newline




\section{Interface : reformat}

mpi run with the **SAME** number of processes that Specfem3D will use. One
file is created by one process for one Spefem3D partion of domain. \newline

copy input\_box.txt and input\_box\_sem\_cart.txt inside the directory where
AxiSEM did the run and add a new parameter file: \newline

reformat.par  \newline
25.         \# output sampling in Hz (time step that will use in Specfem3D simu) \newline
650. 700.   \# begin time and end time (s.) \newline


\section{Running specfem3D}


\section{Set up scripts}


\end{document}
